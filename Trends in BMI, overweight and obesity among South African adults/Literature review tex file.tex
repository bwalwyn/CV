\documentclass[12pt,a4paper]{article}
\usepackage{natbib}
\usepackage{url}
%\usepackage{appendix}
\usepackage[british]{babel}
\usepackage{graphicx}

\bibpunct[:]{(}{)}{;}{a}{,}{,}

\renewcommand{\baselinestretch}{1.5}

\begin{document}

\begin{titlepage}

\begin{center}
{\Huge \bf South African trends in BMI and obesity\\
\bigskip
\Large Literature Review}\\

\today\\
Ben Walwyn (WLWBEN001)\\
{\tt bwalwyn@gmail.com}

\end{center}

\end{titlepage}

\pagenumbering{arabic}

Overweight and obesity is recognised as one of the main causes of death and burden of disease worldwide \citep{ezzati02}. In 2005, 4.8\% of global deaths were attributed to high body-mass index (BMI) \citep{whobook}, when there were around 1.1 billion overweight and more than 300 million obese individuals \citep{haslam05}. Recent estimates \citep{stevens11} show 1.46 billion overweight (34\% of world's population) and over 500 million obese people, and mean BMI growing by about 0.4 kg/m$^2$ per decade since 1980. In North America and Australasia, female BMI rose by 1.2kg/m$^2$ per decade. Effects are seen in all countries, low, middle and high-income \citep{whobook}. High BMI is the 6th leading risk factor globally, accounting for 8\% of disease burden in the North Africa and Middle East region and 10\% in Southern Latin America and Australasia -- the leading risk factor in those regions \citep{gbd}. Data extracted from the 1998 South African health survey estimate mean BMI of 24.1 for males and 28.6 for females, 28\% of the population overweight and 38.6\% of females obese \citep{bradshaw}. 

In order to quantify trends and effects of excess weight, a comparative measure is needed. Body-mass index (BMI) -- originally called the Quetelet Index (kg/m$^2$) after Adolphe Quetelet (1796-1874), who first observed that weight is proportional to the square of height \citep{quetelet} -- is the conventional and most often used anthropometric measure. The index is nearly uncorrelated with height \citep{willett99} and, in this regard, no other measure has been found to be superior. 

The World Health Organisation's expert committee on use of anthropometric measures published a technical report in 1995 \citep{who1995}. It provides a set of guidelines for appropriate measures in epidemiological studies of various life-stages, and the corresponding public health implications in determining risk of mortality and morbidity \citep{onis96}. It finds BMI the measure of choice for adults, with strong identification of at-risk individuals \citep{who1995} and the added advantage of practicality and objectivity \citep{willett99}. However, using BMI may adjust for height, but it cannot differentiate lean and fat mass. Therefore, using it as a measure of fatness assumes that the only cause of weight variation is fat mass \citep{kopelman2000}, ignoring the effect of lean mass. It's also important to bare in mind that BMI values are more informative relative to each other \citep{onis96}, and may not provide sufficient information on an individual basis due to variation of responses to excess weight \citep{kopelman2000}.

The report also provides cut-off values for categorising risk levels of excess weight -- indicating those most at risk and separating those least at risk -- which are mainly used for public communication purposes \citep{stevens11}. The actual selection process was rather arbitrary \citep{who1995}, based on a visual inspection of the BMI and mortality relationship. Most notably, there was a significant rise in all-cause mortality of both sexes above 25kg/m$^2$ \citep{stevens98} and a point of inflection in the curve at 30kg/m$^2$ \citep{who1995}. The WHO set cut-offs at 25, 30 and 40, representing classes 1, 2 and 3 overweight. Class 1 and 2 are referred to in literature as overweight and obesity respectively. 

The study of causes and effects \citep{haslam05,kopelman2000} of excess weight shows that body weight increases when there is a positive energy imbalance between intake and expenditure. Individuals in excess of recommended healthy weights are at higher risk of various forms of cancer, cardiovascular diseases, respiratory disorders, diabetes (type II), osteoarthritis \citep{kopelman2000, haslam05}, reproductive function complications, liver and gall bladder disease \citep{kopelman2007}, stroke, chronic back pain and asthma \citep{guh09}. BMI is also linearly correlated with cholesterol \citep{ezzati05}, and other risk factors \citep{ezzati02}, meaning excess weight acts in association with and through other conditions to increase mortality and morbidity risk.
	
Descriptive statistics provide a useful overview of the size of the risk by country, region or on the global level. Mean BMI is highest in North America and Australasia and lowest in Sub-Saharan Africa \citep{stevens11}. However, the problem in America is blurred by a disproportionate prevalence and mortality in certain population groups \citep{ladabaum2014}. Southern Africa, though included in the low BMI Sub-Saharan Africa region, has a very large obesity rate of 36.4\% in females \citep{stevens12}. This type of distortion is expected for an aggregating statistic and is due to the variation in environmental, genetic and psychosocial factors, relating largely to diet and physical activity, that cause and influence the effect of overweight and obesity \citep{kopelman2000, haslam05}. Some of these factors are measurable \citep{ezzati02} and, hence, many studies have estimated trends in BMI means and overweight and obesity prevalences over time and across the corresponding social interactions. Studies have, therefore, informed on the relationships with these various indicators.

Mean BMI is slightly higher for women than men globally \citep{stevens11}, and much higher for women in South Africa \citep{bradshaw}, indicating a significant independent indicator coefficient. It has also increased more worldwide in women than men (0.5 per decade compared to 0.4) \citep{stevens11}. Globally, more women are obese than men, 297 million (13.8\%) compared to 205 million (9.8\%) \citep{stevens11}. 

The variation between men and women is not constant and this can be seen in most plots with other covariates \citep{ezzati05, stevens12,oecd}. Models are usually fitted and results presented for the two sexes separately. The intercepts for the BMI models fitted to the South African Demographic and Health Survey were 25.79 and 28.04 for mean and women, showing a clear difference. Interactions effects are dependant on the factors -- age, country, population group, income and others. (CITE SOMETHING HERE FROM...)

The relationship between age and BMI tends to be bell-shaped in most populations \citep{oecd}. However, the steepness and the point at which BMI as well as prevalence rate for obesity start to drop varies between countries \citep{stevens11}. Furthermore, this decline in mean BMI may not be an indication of the relationship with BMI, but due to other effects. There are two effects at older ages: excess mortality of high BMI individuals and all-cause illness causing decreased weight prior to death \citep{willett99}. Therefore, the old-age group at any point in time is likely to be comprised of a higher proportion of lower BMI individuals than in another age group. The bell-shape applies to age relationship with obesity, but less so with overweight \citep{oecd}, perhaps reflecting the effect of mortality on those at high risk.

The changes in BMI with age in South Africa appears to be similar for both sexes \citep{bradshaw} and follows the age patterns of most other countries. Adult ages categories for obesity studies typically start at 15, if there are sufficient data. 

BMI means and obesity prevalence rates are also dependant on geographical area. Findings are significantly different between developed and developing countries, mainly by regions \citep{stevens11, stevens12}. OECD countries all report increases, but some more significantly than others \citep{oecd}, indicating regions explain variation over and above any existing relationship to income. Generally, the trend is upward, and male mean BMI has significantly increased in all regions but central Africa and south Asia in recent decades \citep{stevens11}. Female regional trends were flat for central and eastern Europe and central Asia, and increased everywhere else. Female BMI increased in south Asia, however, indicating an interaction effect with sex.

Countries with 50\% or more overweight women increased from 29 countries in 1980 to 101 in 2008. Most of these country were in regions with an existing weight problem \citep{stevens12}. Furthermore, the more substantial part of the increase in prevalence rate occurred in the period from 2000-2008 for developing regions. Some of the regions with no weight gain include high-income countries, for example Japan and Singapore in Asia. This contradicts the disease of affluence and ``Western'' paradigm often associated with obesity \citep{haslam05, prentice06}.

Another reason for the invalid diagnosis of obesity as a symptom of Western culture, is the social and cultural meaning of body shape. Body perception can act as a psychological brake for increasing weight \citep{prentice06}, and where these don't exist, the opposite can also be true. Perceptions amoung many African and Pacific Island populations, which associate fatness with socioeconomic status or beauty \citep{kopelman2000}, breed satisfaction and acceptance of overweight and obesity in middle-aged women and might explain particularly high rates of obesity certain regions.

Data stratified by race in the US, show varying significance of covariates for different groups \citep{ladabaum2014}. Beyond this interaction, racial inequalities still exist in South Africa and so population groups might also be a proxy indicator for socio-economic status, which has a relationship with BMI \citep{ezzati02}. 

Much of the effects of excess weight has for a long time been predominantly experienced by high-income countries, causing 8.4\% of their deaths and 6.4\% of the disability-adjusted life years \citep{ezzati02}. Following this evidence, Ezzati et al. (2005) performed a cross-sectional estimation of the relationship of national income accounting for urbanisation and food share of household expenditure, with age-standardised mean BMI \citep{ezzati05}. As incomes rise, mean BMI increases rapidly to a peak, before falling again. Mean BMI increases most rapidly at an income of about US\$5000, peaks at \$12500 for women and US\$17000 for men, and declines thereafter. Country BMI declines more for women than men at high levels of income. Although the relationship with income is dependent on the country, an income adjusted for purchasing power can give an indication of the point in the transition phase to higher BMI.  

Within country income-BMI relationships show clearly higher relative risk of obesity for lower social classes \citep{oecd}. A similar relationship is seen for education level estimates, with higher risks of overweight and obesity. Both these relationships are much more significant for women than men. Caution must be used when interpreting these results, as they compare countries on a cross-sectional basis, and do not follow countries as incomes rise. 

There are enough data from the US \citep{ezzati05} that indicates an upward shift of the income-BMI relationship over time (1980-2000). This places developing nations on a higher BMI trajectory than anticipated previously, increasing their burden of disease. However, it must be noted that the US was initially excluded from the income-BMI analysis, as it was an outlier with very high mean BMI at a high level of income. The increase may just represent the regional trend identified above, although it seems likely that at least part of globally rising BMI is due to rising income effects.  

Urbanisation and migration are linearly related with rising BMI \citep{ezzati05}. Urban areas are a proxy for \textit{obesogenic} environmental and lifestyle changes that affect energy intake and expenditure \citep{stevens11}, and hence, weight. Urban migration includes the exposure to specific food types and to occupation and transportation changes that require less physical activity \citep{kopelman2000}. Urban obesity in SA (33\%) was higher than non-urban (25\%) in 1998. 

Note that in the context of increasing risk, BMI data need frequent updating. Prevalence rates also become outdated \citep{prentice06}, as using cut-offs for categorising overweight and obesity implies a tipping point -- an increasing mean BMI will approach this point and significant proportions of the population might be newly recognised as overweight or obese. There are, indeed, many regularly performed health surveys which provide data to study trends in BMI and prevalence rates, but their design must be taken into account when considering a statistical model.  





								
Most studies of populations use national survey data, given that it provides sufficient representation. This type of data is cross-sectional, mainly household based, using multi-stage probability sampling stratified by geographical area for selection \citep{oecd}. Using different survey years allows for analysis in time as well \citep{ladabaum2014, stevens11}, although data are often scarce (in time) for developing countries \citep{prentice06}. 
 
There are two major limitations of cross-sectional data. First, design factors of each individual survey can change, like sampling scheme and interview methods. Second, studies do not prospectively follow a cohort of individuals, so in effect individuals are only measured at one point in time. For analysis of certain covariates like diet and physical activity it is difficult to extract accurate data, since there is not constant monitoring and there is only one time-point available to gather the information. 

The most recent US study of 1988-2010 data showed an increase in mean BMI corresponding to an increased proportion of individuals who perform no leisure-time physical activity from 19.1\%-51.7\% for women and 11.4\%-43.5\% for men, while reported caloric intake did not significantly rise within this period \citep{ladabaum2014}. But diet and physical activity in this case are self-reported, and based totally on individual 24-hour recall. Although it is clear physical activity impacted on BMI levels, better design and implementation is needed to estimate to what extent decreases in physical activity versus increases in caloric intake are responsible for rising BMI trends.

Self-reported data are biased, underestimating weight and, therefore, BMI \citep{stevens11}. Comparison of the US health examination and health interview surveys show a clear underestimation of obesity rates for men and more so for women when self-reported \citep{oecd}. Overweight men in this study did not underestimate their weights, however.

Studies of the effect of excess weight must use prospective data \citep{guh09}, but these are often not sufficiently nationally representative and frequent enough to be used in estimating national trends.





	
\textbf{Methods} (SMOOTH FLOW HERE)

Statistical methods chosen to undertake a study are dependant on the purpose of the study and data available.  Age-standardisation is a method required to compare country-mean BMI values since age structures differ between countries and BMI is correlated with age. Generally, a standard population is used to weight the age-categorised mean BMI values for each country, making them comparable   \citep{ezzati05}.

In order to model prevalences, the independent variable, overweight or obesity, must be transformed \citep{stevens12}. A proportion is by definition bounded by 0 and 1, whereas the linear function of explanatory variables is continuous over the whole number line. This transformation is called a logit-link and is defined: $s=log(\frac{p}{1-p})$. 

Covariates can often fluctuate in the short-term, creating a lot of noise. To better determine the underlying relationship with BMI, instead of defining a yearly value of the variable, a weighted average is used to smooth movements changes over time. Income, urbanisation and availability of food types were included as a 10-year average in a recent study \cite{stevens11}. Non-linear relationships can be modelled using higher order polynomials, however, the result may be a fluctuating relationship. In this case, another smoothing technique of using splines in conjunction with redefining the baseline is used to model the relationship of age with BMI \citep{stevens11}. This study included a cubic spline knots at 45 and 60, including new cubic terms when age is greater than 45 and 60 respectively, and shifted the baseline age to 50. In the regression equation, for z being age:

\[\beta(z_i) = \beta_{1}z_i + \beta_{2}z_i^2 + \beta_{3}z_i^3 + \beta_{4}(z_i-45)^3_+ + \beta_{5}(z_i-60)^3_+\]

Multilevel models decompose the variations between individuals and between high-level groups -- households, regional areas or population groups-- whereas using single level models underestimate the standard error unaccounted for in additional variation between groups \citep{multilevel}. The OECD report (2009) outlines a two-level model that measures variation between households and within households. The hierarchy can extend to higher levels, and the statistical model in the most recent global trend study \citep{stevens11}, uses Bayesian and multilevel modelling methods to inform parameter estimates of areas where there was missing data. It does so by nesting country estimates within subregions, regions and the globe, each with non-informative priors. A similar method could be used within a country, nesting provinces within a country and applying non-informative normal priors to each level. 



 linear regression \citep{ladabaum2014}, if there are extensive enough data. 








66\% of non-communicable disease deaths occurred in low-income countries in 2005 \citep{whobook}. Overweight increased by approximately  5\% and 17\% and by 6\% and 5\% per decade for men and women respectively in South Africa from 1980-2008, and has accelerated in the last decade\citep{stevens12}.


A global study of major risk factors \citep{ezzati02} showed that for developing high-mortality countries like South Africa, the burden of disease is dominated by undernutrition and infectious diseases, contributing the most to the loss of life, but as the country develops economically and combats poverty and communicable diseases, it faces new additional risks of mortality and morbidity. 

Income growth, urbanisation and social influences increase the risk of 'double burden' \citep{prentice06}, the risk that additional mortality and morbidity from excess weight, tobacco and alcohol falls on developing countries already dealing with multiple epidemics. 

There is no indication of the epidemic halting. More analysis is needed to identify disparities, understand the dynamics of changing distributions and set targets for stabilisation and intervention in the future. 


\renewcommand{\baselinestretch}{1}

\newpage

\bibliographystyle{dcu(etal)}
\bibliography{Bib(stylemanual)}
\label{ch:bibliography}



\renewcommand{\baselinestretch}{1.5}



\end{document}
